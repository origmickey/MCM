%%%%%%%%%%%%%%%%%%%%%%%%%%%%%%%%%%%%%%%%
%% MCM/ICM LaTeX Template %%
%% 2020 MCM/ICM           %%
%%%%%%%%%%%%%%%%%%%%%%%%%%%%%%%%%%%%%%%%
\documentclass[12pt]{article}
\usepackage{geometry}
\geometry{left=1in,right=0.75in,top=1in,bottom=1in}

%%%%%%%%%%%%%%%%%%%%%%%%%%%%%%%%%%%%%%%%
% Replace ABCDEF in the next line with your chosen problem
% and replace 1111111 with your Team Control Number
\newcommand{\Problem}{B}
\newcommand{\Team}{2012188}
%%%%%%%%%%%%%%%%%%%%%%%%%%%%%%%%%%%%%%%%

\usepackage{newtxtext}
\usepackage{amsmath,amssymb,amsthm}
\usepackage{newtxmath} % must come after amsXXX

%\usepackage[pdftex]{graphicx}
\usepackage{graphicx}

\usepackage{xcolor}
\usepackage{fancyhdr}
\lhead{Team \Team}
\rhead{}
\cfoot{}

\newtheorem{theorem}{Theorem}
\newtheorem{corollary}[theorem]{Corollary}
\newtheorem{lemma}[theorem]{Lemma}
\newtheorem{definition}{Definition}

\title{andrew}

%%%%%%%%%%%%%%%%%%%%%%%%%%%%%%%%
\begin{document}
\graphicspath{{.}}  % Place your graphic files in the same directory as your main document
\DeclareGraphicsExtensions{.pdf, .jpg, .tif, .png}
\thispagestyle{empty}
\vspace*{-16ex}
\centerline{\begin{tabular}{*3{c}}
	\parbox[t]{0.3\linewidth}{\begin{center}\textbf{Problem Chosen}\\ \Large \textcolor{red}{\Problem}\end{center}}
	& \parbox[t]{0.3\linewidth}{\begin{center}\textbf{2020\\ MCM/ICM\\ Summary Sheet}\end{center}}
	& \parbox[t]{0.3\linewidth}{\begin{center}\textbf{Team Control Number}\\ \Large \textcolor{red}{\Team}\end{center}}	\\
	\hline
\end{tabular}}
%%%%%%%%%%% Begin Summary %%%%%%%%%%%
% Enter your summary here replacing the (red) text
% Replace the text from here ...
\begin{center}
	\textbf{Summary}
\end{center}

Wherever there are recreational sandy ocean beaches in the world, there seem to be children (and adults) creating sandcastles on the seashore.So it will be very meaningful and interesting to study strategies to make sandcastles live longer.

\textbf{First},generally, the impact of a single wave on a sand castle is not so strong, so we believe that after each impact on a sand castle, only a portion of the sand at the bottom of the sand castle will be taken away. This will cause a portion of the sand castle above the reduced sand to be suspended.Over time, if the gravity of this part of the sand castle is greater than the attraction from the whole sand castle, then we think that the sand castle has collapsed at this time.In order to determine the relationship between the gravity of the suspended sand castle and the attraction it receives over time and the time required for the gravity of the suspended parts of the sand castle of several basic shapes to exceed the attraction, we establish \textbf{the Sandcastle Collapse Model}. By using the relevant knowledge of unsaturated soil mechanics and geometry, we derived the attraction of suspended sand castles and its gravity, both changing over time. And it can be found that regular quadrangular pyramid the sand castle with a regular quadrangular pyramid survives the longest compared with cube, cylinder, cone and regular triangular pyramid. \textbf{Secondly}, we study the theory of structural mechanics of sand, and we consider various factors that affect the stability of the sand castle. As a result of above, we find that sand pile repose angle is an important mechanical characteristic to characterize sand pile stability, whose main influencing factor is the water content of the sand pile. Thus, using the fitting method with correction coefficients to study the relationship between the repose angle of sandy piles and water content,  we establish \textbf{the Optimal Water-Sand Ratio Model} so as to find the best water-sand ratio to maintain sandcastle stability. \textbf{Thirdly},on the basis of \textbf{the Sandcastle Collapse Model}, we consider the effect of the gravity of the infiltrated rainwater on the stability of the sandcastle during the rainfall. Since that infiltration cannot penetrate the entire sandcastle uniformly, it is believed that rainwater infiltration has no effect on the water-sand ratio of the sandcastle. And considering the rainfall infiltration amount becomes very small when the rainfall intensity is less than the infiltration intensity, so we only study the rainwater infiltration process when the rainfall intensity is greater than the infiltration intensity. Based on this limitation, we establish \textbf{the Rainwater Infiltration Model}. In this model, we combined the empirical formula proposed by Horton et al. To calculate the weight of the sandcastle suspended by the infiltration rainwater.
And the results show that the cube survives the longest of these few geometrie. \textbf{In addition},

What's more, in the sensitivity analysis, we focus on three factors, namely the height of sandcastle, the height of the tide that hits the sandcastle and the depth of the tide-eroded sandcastle,in order to analyze their impacts on the sandcastle stability
% to here
%%%%%%%%%%% End Summary %%%%%%%%%%%

%%%%%%%%%%%%%%%%%%%%%%%%%%%%%%
\clearpage
\pagestyle{fancy}
% Uncomment the next line to generate a Table of Contents
%\tableofcontents 
\newpage
\setcounter{page}{1}
\rhead{Page \thepage\ }
%%%%%%%%%%%%%%%%%%%%%%%%%%%%%%
Begin your paper here


%%%%%%%%%%%%%%%%%%%%%%%%%%%%%%
\end{document}
\end
